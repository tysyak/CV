% A CV template inspired by simple websites with a main infinite scroll body and a little navigation at the side.

\documentclass[lighthipsterblue]{monocolnavbarcv}
% available options are: turquoise, red, lighthipsterblue, rose, blue, pastelgreen,
% as well as: grey, allblack

\usepackage[familydefault,light]{Chivo} %% Option 'familydefault' only if the base font of the document is to be sans serif
% \usepackage[default]{raleway}
% \usepackage[T1]{fontenc}

% \usepackage[defaultsans]{droidsans}
% \usepackage[default]{comfortaa}
% \usepackage{cmbright}
% \usepackage{fetamont}
% \usepackage[default]{gillius}
% \usepackage[light,math]{iwona}
% \usepackage[thin]{roboto}

\getpagesetup{}

% ------------------------------------------------------------------
\title{CV Cristian}
\author{Cristian Romero Andrade}
\date{}

\begin{document}

% ------------------------------------------------

\setupparacol{}
\begin{paracol}{2}

  \switchcolumn{}
  \begin{navbar}
\roundpic{yo2.png}

\getgreyishblackfont


% \bigskip

% \scalebox{0.6}{
%   \iconcross{\Huge}
%   {cvaltcolour}{cvcolour}{\faBook}
%   {\href{mailto:cristian.romero@comunidad.unam.mx}{\faEnvelopeO}}
%   {\href{tel:5531659139}{\faPhone}}
%   {\faCode}
% }

\bigskip
\dotfill{}
\subsection{Habilidades Computacionales}
\begin{minipage}[t]{\onefifthwidth}
  \begin{tabular}{>{\ssmall}r@{\hspace{0.1em}}l}
    \LaTeX{} & \barrule{0.4}{0.25em}{cvcolour} \\
    PHP & \barrule{0.5}{0.25em}{cvcolour} \\
    Python & \barrule{0.5}{0.25em}{cvcolour} \\
    Ruby & \barrule{0.5}{0.25em}{cvcolour} \\
    Elixir/Erlang & \barrule{0.4}{0.25em}{cvcolour} \\
    JS & \barrule{0.5}{0.25em}{cvcolour} \\
    C/C++ & \barrule{0.35}{0.25em}{cvcolour} \\
    Rust & \barrule{0.3}{0.25em}{cvcolour} \\
    GO & \barrule{0.3}{0.25em}{cvcolour} \\
    Shell & \barrule{0.5}{0.25em}{cvcolour}\\
    Docker & \barrule{0.35}{0.25em}{cvcolour} \\
    R  & \barrule{0.1}{0.25em}{cvcolour} \\
    HTML& \barrule{0.48}{0.25em}{cvcolour} \\
  \end{tabular}
\end{minipage}

\medskip{}
\dotfill{}
\subsection{Idioma(s)}
\begin{minipage}[t]{\onefifthwidth}
  \begin{tabular}{>{\ssmall}r@{\hspace{0.1em}}l}
    \textbf{Ingles} & \barrule{0.4}{0.25em}{cvcolour}
  \end{tabular}
\end{minipage}

\end{navbar}

  \begin{makeheader}
  \headername{Cristian}{Romero Andrade}

  \dotfill{}

  \begin{center}
    {
      \footnotesize
      \twitter{\href{https://twitter.com/MascritTrick}{@MascritTrick}}\hfill{}
      \linkedin{\href{https://www.linkedin.com/in/mascrit/}{mascrit}}\hfill{}
      \github{\href{https://github.com/tysyak/}{tysyak}}\hfill{}
      \href{tel:5531659139}{\faPhone~(55) 3165 9139}\hfill{}
      % \faAt~\protect\url{tysyak.xyz}
      \faEnvelope~\href{mailto:cristian.romero@comunidad.unam.mx}{cristian.romero@comunidad.unam.mx}\hfill{}
      \faMapMarker~Coyoacan, Ciudad de México
    }
  \end{center}
\end{makeheader}


  \fancysection{cvcolour}{Edu}{cación}
  \begin{tabular}{l >{\small}p{\paracolwidth} >{\small\itshape\color{cvcolour}}r}
    \faCalendar~2015--2016 & \textbf{Técnico en mantenimiento en micro-cómputo} & CDMX~\faMapMarker \\
                           & CCH Vallejo & \\
    \faCalendar~2016--Actual & \textbf{Ingeniero en Computación} & CDMX~\faMapMarker \\
                           & Facultad de Ingeniería---UNAM & \\
  \end{tabular}
  \medskip

  \fancysection{cvcolour}{For}{talezas}

  {\scriptsize
    \progressarc{1.7mm}{cvcolour}{0.9cm}{cvaltcolour}{\scriptsize \textbf{Progra-}\\\scriptsize{\textbf{mación ágil}}}{}{below}{90} \hfill
    \progressarc{1.7mm}{cvcolour}{0.9cm}{cvaltcolour}{\scriptsize\bf \textbf{Análisis}\\{\scriptsize \textbf{de datos}}}{}{below}{85} \hfill
    \progressarc{1.7mm}{cvcolour}{0.9cm}{cvaltcolour}{\scriptsize\bf \textbf{Gestión de} \\\scriptsize{\textbf{Procesos}}}{}{below}{80}
  }
  \medskip

  \fancysection{cvcolour}{Ex}{periencia}
  \begin{tabular}{r| p{\paracolwidth}}
    \cvevent{\faCalendar~2019--Actual}{Desarrollador de Software}{U.S.E.C.A.D}{Facultad de Ingeniería---UNAM \color{cvaltcolour}}{
    \begin{itemize}
      \item Soporte técnico a equipo e infraestructura de la red interna.
      \item Desarrollo y mantenimiento del sistema de inscripciones.
      \item  Capacidad de llevar a cabo la creación y mantenimiento de
            distintos proyectos de software.
    \end{itemize}
    }\\
    \cvevent{\faCalendar~(2015) julio-agosto}{Técnico en Mantenimiento en Micro-cómputo}{Museo de la Luz}{C.D.M.X. \color{cvaltcolour}}{
    \begin{itemize}
      \item Brinde distintos tipos de mantenimiento a las exposiciones del
            museo al igual a los computadores del personal.
      \item Igualmente lleve la instalación de software y de alternativas
            libres para un mejor desempeño en el trabajo del día a día de mis
            compañeros.
    \end{itemize}
    }

  \end{tabular}

  \medskip

  \fancysection{cvcolour}{Hab}{ilidades}
  \begin{tabular}{>{\small\bfseries}r >{\small}p{\paracolwidth}}
    Desarrollo Back-End: & PHP, Python, Elixir \\
    Sistema de Control de Versiones: & Git, Mercurial \\
    Bases de Datos: & PostgreSQL, mariaDB/MySQL, CouchDB \\
    Sistemas operativos: & Linux (Preferido), Windows. \\
    Desarrollo Front-End: & VueJS, JQuery, Bootstrap, SASS, CSS.\@
  \end{tabular}
\end{paracol}
\end{document}
